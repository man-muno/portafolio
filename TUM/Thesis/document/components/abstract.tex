% Abstract for the TUM report document
% Included by MAIN.TEX


\clearemptydoublepage
\phantomsection
\addcontentsline{toc}{chapter}{Abstract}	

\vspace*{2cm}
\begin{center}
{\Large \bf Abstract}
\end{center}
\vspace{1cm}


In recent years we have witnessed increasing interest in smart environment from researches, and from the public, about smart homes. The appliance industry has noted this interest and also has increased the number of products geared towards home automation.
These products take advantage of wireless communication technologies to monitor, and in some cases autonomously control the environment they are in. Through their interaction and collaboration, they can make spaces more comfortable and more secure. Nevertheless, the alternatives for smart home security are not yet appealing. In order to be attractive to the general consumer, they have to outperform the traditional security systems. 

One of the problems facing traditional home security systems, is the issue of false positives. A false positive, in the context of the home, is defined as reporting an intrusion or a burglary when really nothing has occurred. We developed a system that aims to reduce the occurrence of false positives, while detecting security anomalies. To that effect, we explored the available research and existing algorithms from the networking discipline, which has dealt with intrusion detection since the 1960s. 

We aim to address these issues with the development of Rosie, a system that collects and analyses data from sensors, in order to recognize usual behavior of the inhabitants of the home, and distinguish it from abnormal behavior. 

Rosie, also serves as a platform for studying new intrusion detection algorithms, by providing simple ways of extending the software. As a proof of concept, the system uses algorithms from network intrusion detection systems. These techniques analyze communication, temporal patterns of the network traffic loads, and the content of the packets. The same type of information can be found in smart environments, and is therefore useful for distinguishing usual and unusual behavior of inhabitants.

