
\chapter{System Design}
	In the system design chapter, the previously discussed analysis model is transformed into a system design model. This model includes a clear description of design goals, a subsystem decomposition and strategies for building the system, such as the hardware/software mapping. \cite{Bruegge2004} The following sections describe the system design with strategies and practices of .......   Most of the discussed patterns, strategies and practices can be transferred to similar projects in this application domain.

	\section{Design Goals}
	The intuitive control system design is driven by the following design goals:
	\begin{enumerate}
	\item \textbf{Support of different protocols.} Multiple control vendors for the same fixture type are common among smart spaces, such as the  Intelligent Workplace. Unfortunately, standards are not used consistently and proprietary protocols are still common practice. Therefore, it is necessary to support different control protocols. For instance, an electric light might be controlled by a proprietary protocol of the vendor, while a window blind is controlled by the common BACnet protocol. It is even possible that two lights are addressed by different protocols. The overall system architecture has to deal with these legacy factors that cause problems.
	
	....

	\end{enumerate}
		
	\section{Subsystem Decomposition}


	\section{HW/SW Mapping}

	\section{Persistent Data Management}
	
	\section{Access Control and Security}
	
	
\section{Summary}


\chapter{Object Design}

In the following chapter we refine three application domain concepts in more detail, that is, .... We also provide a precise view of the corresponding solution domain objects and their interfaces to each other. 
...

\section{Custom sections here}
\section{Custom sections here}


		
\section{From the Application Domain to the Solution Domain}

The previous sections approached issues from the application domain. The following section describes the transition to the solution domain model. In object design, the Broker-concept which has been described earlier in the subsystem decomposition, is now refined in more detail. In addition, the object and subsystem interfaces of the broker are specified, off-the-shelf components are selected, and the object model is restructured to attain the discussed design goals.

\subsection{Selection of Programming Language and Off-the-Shelf Components}


\subsection{Interface Specification}



\section{Summary}

In the object design chapter, 

....

. Afterwards, the transformation of the broker's application domain objects to solution domain objects is discussed by specifying object and subsystem interfaces, and selecting off-the-shelf components.
