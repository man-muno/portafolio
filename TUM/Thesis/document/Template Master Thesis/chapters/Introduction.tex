\chapter{Introduction}
\label{chapter:Introduction}

\section{Problem Statement}



\subsection{custom subsections here}



\section{Related Work}
\label{related_work}

Here is an example of how it could look like:

To show the latest research in this field, the following section discusses selected literature and related projects. \\

\textbf{Tsukada and Yasumura, 2004: UBI-FINGER: A simple gesture input device for mobile and ubiquitous environment, \cite{aisl2004-ubi-finger}} \\
Tsukada and Yasumura introduce an interface for mobile environments, called \emph{Ubi-Finger}. The concept enables users to sensuously control various fixtures in the real world using finger gestures. \cite{aisl2004-ubi-finger} The \emph{Ubi-Finger} itself is an input device worn on the fingers. It is connected to a PDA or laptop by a wired serial connection. The target device is defined by infrared sensors attached to each addressable fixture. The actual control commands are transmitted via W-LAN and executed by a server in the background. The gesture recognition is started and stopped using a little button (touch sensor). A target device is then selected by pointing. Afterwards, the device can be controlled with micro-gestures of fingers, like \emph{pushing a switch}, \emph{turning a volume knob}, and so on. \cite{aisl2004-ubi-finger} \\
The authors argue that in the communication of human emotions and wills, the non-verbal means were more important than verbal means. Human gestures are considered to be typical examples of non-verbal communication that help people communicate smoothly. They suggest that "human gestures are very useful communication means, and naturally used by everyone". \cite{aisl2004-ubi-finger} The major benefit of applying gesture input methods to operations of real-world devices is an easy-to-understand mapping of operations with an existing metaphor. \\
Tsukada and Yasumura's ideas of using pointing and gesturing control mechanisms support the ideas of this thesis. However, there are some disadvantages in the proposed approach. First, the approach requires the installation of additional infrastructure (infrared sensors) on each addressable fixture. Apart from the visual detraction, this solution implies additional costs for the electronic pieces and labor for its installation. Second, the prototype finger is still too big and inconvenient for daily work. The fact that it has to be wired to a PDA might also be distracting for a number of users. \\

\section{Terminology}

The following section introduces definitions of ambiguous words that are used continuously throughout this thesis. 

\begin{itemize}

\item \textbf{Fixture.} A fixture is an instance of a specific fixture type (see below). An instrumented space consists of a number of fixtures that affect the environmental conditions of the occupant. 

\item \textbf{Fixture type.} A fixture type is the generalization of a fixture. Fixture types in the Intelligent Workplace include:
	\begin{itemize}
		\item Light
		\item Addressable plug load
		\item Window blind
		\item External louvers
		\item Operable window
		\item Coolwave
	\end{itemize}

\item \textbf{Instrumented space.} An instrumented space is an indoor environment with addressable fixtures

\item \textbf{Smart space.} The word "smart space" is used interchangeably with instrumented space.

\item \textbf{Mobile device.} A mobile device is the user's smart phone or similar device. The developed prototype of this project works with an iPhone as mobile (control-) device.

\item \textbf{Occupant.} The occupant is the user of an instrumented space. The overall project purpose is to provide well-being and comfort to the occupant while reducing the consumed energy.

\item \textbf{User.} The word "user" is used interchangeably with occupant.

\item \textbf{Facility Manager.} The facility manager is the responsible person for maintenance, care and functionality of a building. This thesis assumes that the facility manager is the administrator of the proposed intuitive controller and cares for the initial set up of the system. Especially in non-commercial smart home environments, it has to be considered that the roles of the building owner, occupant and facility manager are actually taken by the same person.

\end{itemize}
