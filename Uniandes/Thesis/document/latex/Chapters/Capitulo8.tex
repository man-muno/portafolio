% Chapter 8

\chapter{Conclusiones y Trabajo Futuro} % Write in your own chapter title
\label{Chapter8}
\lhead{Chapter 8. \emph{Conclusiones y Trabajo Futuro}} % Write in your own chapter title to set the page header


\section{Conclusiones}
A lo largo de este trabajo se present� uno de los problemas, que teniendo en cuenta la cantidad de herramientas que lo soportan, ha sido muy poco explorado. Las preocupaciones transversales y la modularizaci�n de ellas son una cara de las aplicaciones orientadas a \textit{workflow} que presenta una gran importancia debido a las ventajas que representa contar con mecanismos que permitan tanto definir nuevas funcionalidades sobre procesos como encapsularlas.

El enfoque de este trabajo de tesis fue la construcci�n de un motor de BPEL que no solo permitiera la definici�n de nuevos comportamientos encapsulados, sino tambi�n una aproximaci�n para resolver el problema de interferencia entre aspectos, el cual afecta a los lenguajes orientados por aspectos.

Gracias a que se hizo uso de las propuestas realizadas por el proyecto Cumbia, es posible tener un motor de BPEL funcional y  a su vez realizar la f�cil integraci�n con otro modelo que representa los comportamientos transversales, sin perder la clara diferenciaci�n entre los elementos que hacen parte de las preocupaciones transversales y los elementos del proceso, otorg�ndole a los comportamientos transversales identidades de primer orden.

Debido a que se utilizaron objetos abiertos, es posible hacer un f�cil monitoreo de los aspectos que est�n siendo ejecutados en cierto momento en el tiempo o conocer f�cilmente cuales son los aspectos que afectan directamente instancias de proceso especificas. Adem�s, gracias a la flexibilidad intr�nseca de utilizar modelos ejecutables extensibles, es posible cambiar f�cilmente las implementaciones aqu� propuestas, de tal manera que se acomoden a las necesidades especificas de diferentes contextos a muy bajo costo. Un ejemplo claro de esto es la posibilidad de definir nuevos \textit{transition points}, donde la manera de ordenar los \textit{advices} sea una estructura que se acomode mejor a las necesidades del negocio, en vez de un grafo dirigido.

Es posible dise�ar componentes extra que permitan enriquecer al motor de BPEL que ayuden a manejar requerimientos no funcionales, como la transaccionalidad de los procesos o la seguridad en el intercambio de mensajes. A su vez, la f�cil extensi�n de los motores, permite desarrollar componentes para contextos donde la l�gica del negocio est� definida usando reglas de negocio.

\section{Trabajo Futuro}
Como parte del trabajo futuro se propone realizar la misma experimentaci�n sobre otros de los activos existentes del proyecto, como por ejemplo el motor de BPMN.

Tambi�n se propone hacer una extensi�n sobre el lenguaje de puntos de corte para tener en cuenta puntos que representan, por ejemplo, cuando una actividad x es ejecutada despu�s de una actividad z o involucrar otros dominios, c�mo el de recursos, para tener expresiones que puedan definir puntos de corte donde cierto participante ejecute cierta actividad. Tambi�n puntos de corte donde se monitoree cuando una variable es modificada o le�da.

Otra propuesta es hacer algo similar a lo que hace \textit{Padus}, que permite definir \textit{advices} de tipo \textit{in}, los cuales son colocados dentro de una actividad especifica, por ejemplo dentro de un \textit{flow}. Tambi�n es posible tener en cuenta otros tipos de \textit{advices} que son utilizados en \textit{AO4BPEL} que pueden ser ejecutados en paralelo al elemento sobre los cuales est�n definidos.
